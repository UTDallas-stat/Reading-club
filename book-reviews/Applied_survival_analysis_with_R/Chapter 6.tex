\documentclass{beamer}

% Some common packages
\usepackage{graphicx, color}
\usepackage{alltt}
\usepackage{booktabs, calc, rotating}
\usepackage[round]{natbib}
\usepackage{multicol}
\usepackage{amsmath, amsbsy, amssymb, amsthm, graphicx}
\usepackage[english]{babel}
\usepackage{xkeyval} 
\usepackage{xfrac}
\usepackage[normalem]{ulem}
\usepackage{fancyvrb} 
\usepackage{tikz, geometry, tkz-graph, xcolor}
\usepackage[latin1]{inputenc}
\usepackage{times}
\usepackage[T1]{fontenc}

% Shortcuts
\newcommand{\empr}[1]{{\emph{\color{red}#1}}}
\newcommand{\cov}{\mathrm{cov}}
\newcommand{\pkg}[1]{{\textbf{\texttt{#1}}}}
\newcommand{\dif}{\mathrm{d}}
\newcommand{\bigbrk}{\vspace*{2in}}
\newcommand{\smallbrk}{\vspace*{.1in}}
\newcommand{\midbrk}{\vspace*{1in}}
\newcommand{\red}[1]{{\color{red}#1}}
\newcommand{\blue}[1]{{\color{blue}#1}}
\newcommand{\green}[1]{{\color{green}#1}}
\newcommand{\calc}[1]{{\fbox{\mbox{#1}}}}
\newcommand{\Var}{\mathrm{Var}}%
\newcommand{\Cov}{\mathrm{Cov}}%

\mode<presentation>
{
	\usetheme{UTD}
	\usecolortheme[RGB={200,0,0}]{structure}
	\setbeamercovered{transparent}
}

% fancy for Verbatim?
\fvset{frame=single,framesep=1mm,fontfamily=courier,fontsize=\scriptsize,numbers=left,framerule=.3mm,numbersep=1mm,commandchars=\\\{\}}


\title[Survival Analysis]{Applied Survival Analysis Using R\\ Chapter 6: Model Selection and Interpretation}
\author[Qi Guo]{Qi Guo}
\institute[UTD]{Department of Mathematical Sciences \\ 
	The University of Texas at Dallas}
\date{April, 13 2019}
	
\begin{document}

\begin{frame}
  \titlepage
\end{frame}

% Set up UTD backgroud
\setbeamercolor*{item}{fg=red}
\bgroup
\usebackgroundtemplate{
\tikz[overlay,remember picture] \node[opacity=0.05, at=(current page.center)] {
   \includegraphics[height=\paperheight,width=\paperwidth]{UTDbg}};}


\section[Outline]{}
\begin{frame}
  \tableofcontents
\end{frame}

\section{Covariate Adjustment}
\begin{frame}[fragile]
\frametitle{Covariate Adjustment}
\begin{itemize}
\item Most of our study is a randomized clinical trial, the focus will be on \empr{comparing the effectiveness of different treatments}.
\item A successful randomization procedure should ensure that \empr{confounding covariates are balanced} between the treatments.
\item {\color{red}Goal}:Use methods to {\color{red}sift through} a potentially large number of potential explanatory variables to find the important ones.
\item Example(\texttt{coxph} model of the effect of treatment on survival unadjusted for the genetic mutation status of the patients)
\end{itemize}
\begin{Verbatim}
> coxph(Surv(ttAll, status) ~ trt)
     coef    exp(coef)    se(coef)    z        p 
trt 0.464       1.59       0.117    3.96    7.6e-05

Likelihood ratio test=15.5 on 1 df,p=8.2e-05 
\end{Verbatim}
\end{frame}

\pagebreak
\begin{frame}[fragile]
\frametitle{Conclusion and stratify}
\begin{itemize}
\item Conclusion:
\begin{itemize} 
\item The estimate of the log hazard ratio treatment effect, $\hat{\beta}$, is 0.464, so higher hazards are associated with the treatment than with the control, unfortunate result.
\item The value of $e^{\hat{\beta}}=1.59$, suggesting (incorrectly, as we know) that the treatment is associated with a 59\% additional risk of death over the control.
\end{itemize}
\item Stratify on \pkg{genotype}:
\begin{Verbatim}
> coxph(Surv(ttAll, status) ~ trt + strata(genotype))
       coef    exp(coef)    se(coef)    z        p 
trt  -0.453      0.636       0.164    -2.76    0.0058

Likelihood ratio test=7.66 on 1 df,p=0.00566 
\end{Verbatim}
\item Conclusion: The coefficient is negative, indicating that, within each genotype, \empr{the treatment is effective}.
\end{itemize}
\end{frame}

\pagebreak
\begin{frame}[fragile]
\frametitle{Explicitly estimating the genetic effect}
\begin{Verbatim}
> coxph(Surv(ttAll, status) ~ trt + genotype)
                coef      exp(coef)    se(coef)    z        p 
trt            -0.453       0.636        0.164    -2.76    0.0058
genotypewt     -1.568       0.209        0.183    -8.59    0.0000
Likelihood ratio test=93.4 on 2 df, p=0
\end{Verbatim}
\begin{itemize}
\item Conclusion:The \pkg{wild type genotype} has lower hazard than the \pkg{reference (mutant) genotype}, and thus that the mutant genotype incurs additional risk of death.
\end{itemize} 
\end{frame}


\section{Categorical and Continuous Covariates}
\begin{frame}
\frametitle{Indicator or Dummy variable}
\begin{itemize}
\item The previous sections considered a partial likelihood for comparing two groups, indexed by a covariate $z$. Since $z$ can take the values 0 or 1 depending on which of two groups a subject belongs to, this covariate is called an \texttt{indicator} or \texttt{dummy variable}.
\item When \empr{categorical variables} with three or more variables, we will need \texttt{multiple dummy variables}. For example:
\begin{itemize}
\item If a research question is how survival in \texttt{non-white} groups compares to survival in \texttt{whites}, one would select ``\texttt{white}'' as the reference variable. Since there are four levels, we need to create {\color{red}three} dummy variables, say, $z_2$, $z_3$, and $z_4$ to represent ``\texttt{race}''. Then for a \texttt{white} patient, all three would take the value zero. For an Asian person, we would have $z_2=1$, and $z_1=z_3=0$.
\end{itemize}
\end{itemize}
\end{frame}

\pagebreak
\begin{frame}
\frametitle{\texttt{k} covariates model and enhance}	
\begin{itemize}
\item \texttt{k} covariates model:
\begin{equation}
\log(\psi_i) = z_{1i}\beta_1 + z_{2i}\beta_2 + \cdots + z_{ki}\beta_k. 
\end{equation}
\item For each covariate, the parameter $\beta_j$ is the log hazard ratio for the effect of that parameter on survival, adjusting for the other covariates.
\item \texttt{Matrix} form: $\log(\psi_i)=z_{i}^{'}\beta$ (for Patient i), where $z_{i}^{'}$ (the transpose of $z_i$) is a $1\ x\ k$ matrix (i.e. a row matrix) of covariates, and $\beta$ is a $k\ x\ 1$ matrix (i.e. a column matrix) of parameters.
\item {\color{red}Enhance} the model: 
\begin{itemize}
\item {\color{blue}$\otimes$} If a continuous variable is not linearly related to the log hazard, transform it using, eg: a \texttt{logarithmic} or \texttt{square root function}.
\item  {\color{purple}$\otimes$} ``\texttt{discretize}'' a variable, eg: split the ``\texttt{age}'' into three pieces, ``\texttt{under 50}'' and ``\texttt{50-64}'', and ``\texttt{65 and above}'' and entered into the model as a categorical variable.
\end{itemize} 
\end{itemize} 
\end{frame}

\pagebreak
\begin{frame}
\frametitle{Difference}	
\begin{itemize}
\item {\color{purple}$\odot$} But it is incorporate for \empr{interaction} terms.
\item Difference with linear and logistic regression model:
\begin{itemize}
\item Survival data can evolve over time, there is a possibility that some covariate values may also \empr{change as time passes}.
eg:Time-related variables like \texttt{age} must also be defined and fixed by taking their value at the beginning of the trial, even though patients will age as the trial progresses(Chapter 8).
\item There is {\color{red}no intercept term} in proportional hazards models, if there were one, it would be absorbed into the baseline hazard(canceled out in \texttt{num} and \texttt{den}).

\end{itemize} 
\end{itemize} 
\end{frame}


\pagebreak
\begin{frame}[fragile]
\frametitle{Example}	
\begin{problock}{Tidy the data}
Suppose that we have two black patients,two white patients,and two  patients of other races, with ages 48, 52, 87, 82, 67, and 53, respectively. We may enter these data values as follows:
\end{problock}
\begin{Verbatim}
> race <- factor(c("black", "black", "white", "white", "other", "other"))
age <- c(48, 52, 87, 82, 67, 53)
\end{Verbatim}
\begin{itemize}
\item Create matrix using ``\texttt{model.matrix}'' function(In my \texttt{R for data science} presentation)
\end{itemize}
\begin{Verbatim}
> model.matrix(~ race + age)[,-1]
     raceother     racewhite   age
1        0              0       48
2        0              0       52
3        0              1       87
4        0              1       82
5        1              0       67
6        1              0       53
\end{Verbatim}
\end{frame}

\pagebreak
\begin{frame}[fragile]
\frametitle{Example}	
\begin{itemize}
\item If we need to use \texttt{whites} as the reference, we can change the race factor to have ``\texttt{whites}'' as the reference level
\end{itemize}
\begin{Verbatim}
> race <- relevel(race, ref="white")
> model.matrix(~ race + age)[,-1]
     raceblack     raceother   age
1        1              0       48
2        1              0       52
3        0              0       87
4        0              0       82
5        0              1       67
6        0              1       53
\end{Verbatim}
\begin{itemize}
\item Three covariates, say, $z_1$, $z_2$, and $z_3$, the first two of which are \texttt{dummy variables} for black race and other race, and the third a \texttt{continuous variable}, age.
\item For a black 48-year old person, the {\color{red}log hazard ratio} is:
\begin{equation}
\log(\psi_1) = z_{11}\beta_1 + z_{12}\beta_2 +z_{13}\beta_3 = 1 x \beta_1 + 0 x \beta_2 + 48 x \beta_3. 
\end{equation}
\end{itemize}
\end{frame}

\pagebreak
\begin{frame}[fragile]
\frametitle{Example}	
\begin{itemize}
\item $\beta_1$ represents the log hazard ratio for blacks as compared to whites, and $\beta_3$ represents the change in log hazard ratio that would correspond to a one-year change in age.
\item Add {\color{red}Interaction}:
\begin{Verbatim}
> model.matrix(~ race + age + race:age)[,-1]
    raceblack   raceother   age   raceblack:age  raceother:age
1        1           0      48         48             0
2        1           0      52         52             0
3        0           0      87         0              0
4        0           0      82         0              0
5        0           1      67         0             67
6        0           1      53         0             53
\end{Verbatim}
\end{itemize}
\end{frame}


\pagebreak
\begin{frame}[fragile]
\frametitle{Example}	
\begin{problock}{Simulation}
Generate a small survival data set and show how models are incorporated into a survival problem
\end{problock}
\begin{itemize}
\item Generate 60 ages between 40 and 80 at random and categorize,and make ``\text{white}'' the reference category:
\begin{Verbatim}
> age <- runif(n=60, min=40, max=80)
> race <- factor(c(rep("white", 20), rep("black", 20), 
rep("other", 20)))
> race <- relevel(race, ref="white")
\end{Verbatim}
\item The variables are exponentially distributed with a particular rate parameter that depends on the covariates, specified the log rate parameter to have baseline -4.5, and ``\texttt{age}'' increase the log rate by 0.05 per year:
\begin{Verbatim}
> log.rate.vec <- -4.5 + c(rep(0,20), rep(1,20), rep(2,20)) 
+ age*0.05
\end{Verbatim}
\end{itemize}
\end{frame}


\pagebreak
\begin{frame}[fragile]
\frametitle{Example}	
\begin{itemize}
\item No censoring
\begin{Verbatim}
>tt <- rexp(n=60, rate=exp(log.rate.vec)) 
>status <- rep(1, 60)
\end{Verbatim}
\item Fit a \empr{Cox proportional hazards model}
\begin{Verbatim}
> library(survival)
> result.cox <- coxph(Surv(tt, status) ~ race + age)
> summary(result.cox)
n= 60, number of events= 60
            coef    exp(coef)  se(coef)    z     Pr(>|z|)
raceblack  1.15154   3.16305   0.36752   3.133    0.00173 **
raceother  2.49905  12.17087   0.42936   5.820   5.87e-09 ***
age        0.07798   1.08110   0.01448   5.385   7.24e-08 ***
---
Signif. codes:  0 ***  0.001  **  0.01 * 0.05 . 0.1     1
\end{Verbatim}
\item Conclusion:
\begin{itemize} 
\item The coefficient estimates, 1.15, 2.50, and 0.08, are close to the true values from the simulation, (1, 2, and 0.05).
\item The  \texttt{blacks} have 3.16 times the risk of death as do \texttt{whites}.
\end{itemize}
\end{itemize}
\end{frame}

\section{Nested Models}
\begin{frame}[fragile]
\frametitle{Nested Models}
\begin{defblock}{Nested Models}
When comparing two models, the covariates of one model must be a {\color{red}subset} of the covariates in the other.
\end{defblock}
\begin{itemize}
\item Example:\linebreak 
Model A: ageGroup4\linebreak
Model B: employment\linebreak
Model C: ageGroup4 + employment\linebreak
Here, Model A is nested in Model C, and Model B is also nested in Model C, so these models can be compared using statistical tests
\end{itemize}
\begin{Verbatim}
> levels(ageGroup4)
[1] "21-34" "35-49" "50-64" "65+"
> levels(employment) # "ft" refers to full-time, "pt" is part-time
[1] "ft" "other" "pt"  
\end{Verbatim}
\end{frame}


\pagebreak
\begin{frame}[fragile]
\frametitle{\texttt{Coxph} model}	
\begin{itemize}
\item Model A:
\begin{Verbatim}
> modelA.coxph <- coxph(Surv(ttr, relapse) ~ ageGroup4)
> modelA.coxph
                  coef    exp(coef)   se(coef)     z        p 
ageGroup435-49   -0.453     0.636       0.164    -2.76    0.0058
ageGroup450-64   -1.568     0.209       0.183    -8.59    0.0000
ageGroup465+     -0.3173    0.728       0.444   -0.7153   0.470

Likelihood ratio test=12.2 on 3 df, p=0.00666
\end{Verbatim}
\item Model B:
\begin{Verbatim}
> modelB.coxph <- coxph(Surv(ttr, relapse) ~ employment)
> modelB.coxph
                  coef     exp(coef)   se(coef)     z       p 
employmentother  -0.453      0.636       0.164    -2.76   0.0058
employmentpt     -1.568      0.209       0.183    -8.59   0.0000

Likelihood ratio test=2.06 on 2 df, p=0.357
\end{Verbatim}
\end{itemize}
\end{frame}

\pagebreak
\begin{frame}[fragile]
\frametitle{\texttt{Coxph} model}	
\begin{itemize}
\item Model C:
\begin{Verbatim}
> modelC.coxph <- coxph(Surv(ttr, relapse) ~ ageGroup4 +
employment)
> modelC.coxph
                  coef    exp(coef)   se(coef)      z        p 
ageGroup435-49   -0.130     0.878       0.321    -0.404   0.6900
ageGroup450-64   -1.024     0.359       0.359    -2.856   0.0043
ageGroup465+     -0.782     0.457       0.505    -1.551   0.1200
employmentother   0.526     1.692       0.275     1.913   0.0560
employmentpt      0.500     1.649       0.332     1.508   0.1300

Likelihood ratio test=16.8 on 5 df, p=0.00492
\end{Verbatim}
\item The \texttt{log-likelihoods}:
\begin{Verbatim}
> logLik(modelA.coxph) 
'log Lik.'  -380.043 (df=3)
> logLik(modelB.coxph)
'log Lik.' -385.1232 (df=2)
> logLik(modelC.coxph)
'log Lik.' -377.7597 (df=5)
\end{Verbatim}
\end{itemize}
\end{frame}

\pagebreak
\begin{frame}[fragile]
\frametitle{Compare}	
\begin{itemize}
\item Determining if ``\texttt{employment}'' belongs in the model by comparing Models A and C, the hypothesis test:
\begin{itemize}
\item $H_0$: The three coefficients for ``\texttt{employment}''  are zero. \linebreak
$H_A$: Not all zero.
\end{itemize}
\item The likelihood ratio statistic is: 
\begin{equation}
2(\ell(\hat{\beta}_{full})-\ell(\hat{\beta}_{reduced})) = 2(-377.7597+380.043) = 4.567
\end{equation}
\item Compare this to a chi-square distribution with $5 -3 = 2$ degrees of freedom.
\begin{Verbatim}
> pchisq(4.567, df=2, lower.tail=F)
[1] 0.1019268
\end{Verbatim}
\item Conclusion: The effect of ``\texttt{employment}'' is not statistically significant when ``\texttt{ageGroup4}'' in the model.
\end{itemize}
\end{frame}


\pagebreak
\begin{frame}[fragile]
\frametitle{ANOVA}	
\begin{itemize}
\item In our \pkg{STAT 6337:Advanced Statistic Method I} class, we use ``\texttt{anova}'' function, which is more direct.
\begin{Verbatim}
> anova(modelA.coxph, modelC.coxph)
Analysis of Deviance Table
Cox model: response is Surv(ttr, relapse)
Model 1: ~ ageGroup4
Model 2: ~ ageGroup4 + employment
   loglik   Chisq    Df   P(>|Chi|)
1 -380.04
2 -377.76   4.5666    2   0.1019
\end{Verbatim}
\end{itemize}
\end{frame}

\section{The Akaike Information Criterion for Comparing Non-nested Models}
\begin{frame}[fragile]
\frametitle{AIC}
\begin{itemize}
\item The Akaike Information Criterion, or \empr{AIC}:
\begin{equation}
AIC = -2\cdot \ell(\hat\beta) + 2\cdot k
\end{equation}
\item where $\ell(\hat\beta)$ denotes the value of the partial log likelihood at the M.P.L.E. for a particular model, and $k$ is the number of parameters in the model.
\item The AIC balances two quantities which are properties of a model:
\begin{itemize}
\item {\color{red}Goodness of fit}: $-2\cdot \ell(\hat\beta)$,this quantity is smaller for models that fit the data well
\item The number of parameters is a measure of complexity.
\end{itemize}
\item Conclusion: Smaller is better.
\end{itemize}
\end{frame}


\pagebreak
\begin{frame}[fragile]
\frametitle{AIC}	
\begin{itemize}
\item Compute the AIC for model A:\linebreak
$AIC = 2 x 380.043 + 2 x 3 = 766.086$
\item Use the ``\texttt{AIC}'' function:
\begin{Verbatim}
> AIC(modelA.coxph)
[1] 766.086
> AIC(modelB.coxph)
[1] 774.2464
> AIC(modelC.coxph)
[1] 765.5194
\end{Verbatim}
\item Conclusion: The best fitting model from among these three, using the AIC criterion, is Model C.
\end{itemize}
\end{frame}

\pagebreak
\begin{frame}[fragile]
\frametitle{stepwise procedure}	
\begin{itemize}
\item Using the AIC criterion:
\begin{Verbatim}
> modelAll.coxph <- coxph(Surv(ttr, relapse) ~ grp + gender + 
race + employment + yearsSmoking + levelSmoking +
ageGroup4 + priorAttempts + longestNoSmoke)
> result.step <- step(modelAll.coxph, scope=list(upper=~ grp + 
gender + race + employment + yearsSmoking +
levelSmoking + ageGroup4 + priorAttempts + longestNoSmoke, 
lower=~grp) )
\end{Verbatim}
\begin{Verbatim}
Start: AIC=770.2 Surv(ttr, relapse) ~ grp + gender + race + 
employment + yearsSmoking + levelSmoking + ageGroup4 + 
priorAttempts + longestNoSmoke
                   Df        AIC
- race              3      766.98
- yearsSmoking      1      768.20
- gender            1      768.20
- priorAttempts     1      768.24
- levelSmoking      1      768.47
- longestNoSmoke    1      769.04
<none>                     770.20
- employment        2      772.45
- ageGroup4         3      774.11
\end{Verbatim}
\end{itemize}
\end{frame}

\pagebreak
\begin{frame}[fragile]
\frametitle{stepwise procedure}	
\begin{itemize}
\item Conclusion:The terms ordered from the one which, when deleted, yields the greatest AIC reduction (``\texttt{race}'' in this case) to the smallest reduction (``\texttt{ageGroup4}''). Thus, ``\texttt{race}'' is deleted.
\item Last step:
\begin{Verbatim}
Step: AIC=758.42 Surv(ttr, relapse) ~ grp + employment 
+ ageGroup4
                          Df        AIC
<none>                             758.42
+ longestNoSmoke           1       759.10
- employment               2       760.31
+ yearsSmoking             1       760.34
+ gender                   1       760.39
+ priorAttempts            1       760.40
+ levelSmoking             1       760.41
+ race                     3       761.53
- ageGroup4                3       767.24
\end{Verbatim}
\item The ``\texttt{+}'' sign shows the effect on AIC of adding certain terms. This table shows that no addition or subtraction of terms results in further reduction of the AIC.
\end{itemize}
\end{frame}


\pagebreak
\begin{frame}
\frametitle{BIC}	
\begin{itemize}
\item An alternative to the AIC is the ``\texttt{Bayesian Information Criterion}'', sometimes called the ``\texttt{Schwartz criterion}'', \empr{BIC} is given by:
\begin{equation}
BIC = -2\cdot \log(L) + k\cdot \log(n)
\end{equation}
\item BIC penalizes the number of parameters by a factor of {\color{red}$\log(n)$} rather than by a factor of 2 as in the AIC.
\item The BIC in model selection will tend to result in models with {\color{red}fewer} parameters as compared to AIC.
\end{itemize}
\end{frame}
\end{document}
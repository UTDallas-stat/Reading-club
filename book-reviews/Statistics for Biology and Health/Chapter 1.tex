\documentclass{beamer}

% Some common packages
\usepackage{graphicx, color}
\usepackage{alltt}
\usepackage{booktabs, calc, rotating}
\usepackage{natbib}
\usepackage{multicol}
\usepackage{amsmath, amsbsy, amssymb, amsthm, graphicx}
\usepackage[english]{babel}
\usepackage{xkeyval} 
\usepackage{xfrac}
\usepackage[normalem]{ulem}
\usepackage{fancyvrb} 
\usepackage{tikz, geometry, tkz-graph, xcolor}
\usepackage[latin1]{inputenc}
\usepackage{times}
\usepackage[T1]{fontenc}

% Some shortcuts
\newcommand{\cov}{\mathrm{cov}}
\newcommand{\dif}{\mathrm{d}}
\newcommand{\bigbrk}{\vspace*{2in}}
\newcommand{\smallbrk}{\vspace*{.1in}}
\newcommand{\midbrk}{\vspace*{1in}}
\newcommand{\red}[1]{{\color{red}#1}}
\newcommand{\blue}[1]{{\color{blue}#1}}
\newcommand{\green}[1]{{\color{green}#1}}
\newcommand{\calc}[1]{{\fbox{\mbox{#1}}}}
\newcommand{\Var}{\mathrm{Var}}%
\newcommand{\Cov}{\mathrm{Cov}}%

\usetheme[numbering=fraction, progressbar = frametitle]{metropolis}
% section page?
% \usetheme[sectionpage=none]{metropolis}
% place progressbar under frametitle?
% \usetheme[progressbar = frametitle]{metropolis}



% To center title?
\setbeamertemplate{title page}[default]
% To add a rectangle box for title?
% \setbeamercolor{titlelike}{parent = frametitle}   

% item color and shape
% \setbeamertemplate{itemize item}{\color{mDarkTeal}$\bullet$}
% \setbeamertemplate{itemize subitem}{\color{mDarkTeal}$\bullet$}
% \setbeamertemplate{itemize subsubitem}{\color{mDarkTeal}$\bullet$}

% UTD colors
\definecolor{warmgray10}{RGB}{118,106,98}
\definecolor{warmgray2}{RGB}{213,210,202}
\definecolor{warmgray1}{RGB}{230,230,230}
\definecolor{flameOrange}{RGB}{199,91,18}
\definecolor{ecoGreen}{RGB}{0,133,86}

% only change item color 
\setbeamercolor*{itemize item}{fg = mDarkTeal}
\setbeamercolor*{itemize subitem}{fg = mDarkTeal}
\setbeamercolor*{itemize subsubitem}{fg = mDarkTeal}
% enumerate color
\setbeamercolor*{enumerate item}{fg = mDarkTeal}
\setbeamercolor*{enumerate subitem}{fg = mDarkTeal}
\setbeamercolor*{enumerate subsubitem}{fg = mDarkTeal}
% description color
\setbeamercolor*{description item}{fg = mDarkTeal}
\setbeamercolor*{description subitem}{fg = mDarkTeal}
\setbeamercolor*{description subsubitem}{fg = mDarkTeal}

% define blocks
\newenvironment<>{problock}[1]{
  \begin{actionenv}#2
    \def\insertblocktitle{#1}
    \par
    \mode<presentation>{
      \setbeamercolor{block title}{fg = white, bg = ecoGreen} %mDarkTeal}
      \setbeamercolor{block body}{fg = black, bg = warmgray1}
    }
    \usebeamertemplate{block begin}}
  {\par\usebeamertemplate{block end}\end{actionenv}}

\newenvironment<>{defblock}[1]{
  \begin{actionenv}#2
    \def\insertblocktitle{#1}
    \par
    \mode<presentation>{
      \setbeamercolor{block title}{fg = white,bg = flameOrange} %mDarkBrown}
      \setbeamercolor{block body}{fg = black,bg = warmgray1}
    }
    \usebeamertemplate{block begin}}
  {\par\usebeamertemplate{block end}\end{actionenv}}

% \setbeamerfont{block title}{size={}}
% \setbeamercolor{titlelike}{parent=structure,bg=flameOrange!80!black,fg=white} % title
\mode
<all>

% fancy for Verbatim?
\fvset{frame = single, framesep = 1mm,fontfamily = courier,
  fontsize = \scriptsize, numbers = left, framerule = .3mm,
  numbersep = 1mm,commandchars = \\\{\}}

% reset text width
\setbeamersize{text margin left = 5pt,text margin right = 5pt}

\title[Short title]{Statistics for Biology and Health\\ Chapter 1 Basic Quantities and Models \\ 
}
\author[Author name]{Qi Guo}
% \institute[UTD]{Department of Mathematical Sciences \\ The University of Texas at Dallas}
\date{July 18,2019}
% UTD logo on title page
% \titlegraphic{\includegraphics[width=5cm]{mono_nsm_print_header.jpg}}
\titlegraphic{\includegraphics[scale = .27]{UTDmono_NSM_header_RGB-1.png}}

\begin{document}

% default title thicker line
% need to adjust
% \frame{
% \maketitle
% \begin{tikzpicture}[remember picture, overlay]
%   \def\xmargin{.1cm}     % Left and right margin of the title line
%   \def\yshift{-0.1cm}   % Moves the title line upwards (-1.25cm moves title line downwards)
%   \draw[mLightBrown,line width=2pt] (current page.west) ++(\xmargin,\yshift) -- ++({\paperwidth-2*\xmargin},0);
% \end{tikzpicture}
% }

\frame{
  \vspace{1cm}
  \maketitle
  \begin{tikzpicture}[remember picture, overlay]
    \def\xmargin{.8cm}     % Left and right margin of the title line
    \def\yshift{0.6cm}   % Moves the title line upwards 
    \draw[mLightBrown,line width=2.2pt] (current page.west) ++(\xmargin,\yshift) -- ++({\paperwidth-2*\xmargin},0);
  \end{tikzpicture}
}


% Set up UTD backgroud
% Show background from here on
\setbeamercolor*{item}{fg=red}
\bgroup
\usebackgroundtemplate{
  \tikz[overlay,remember picture] \node[opacity=0.05, at=(current page.center)] {
    \includegraphics[height=\paperheight,width=\paperwidth]{UTDbg}};}


\begin{frame}{Contents}
  \setbeamertemplate{section in toc}[sections numbered]
  \tableofcontents[hideallsubsections]
\end{frame}

\section{Introduction}
\begin{frame}{Survival data}
\begin{itemize}
\item A common feature of survival data sets is they contain either censored or truncated observations.
\item Censored data arises when an individual's life length is known to occur only in a certain period of time.
\begin{enumerate}
\item Right censoring is known that the individual is still alive at a given time.
\item Left censoring when all that is known is that the individual has experienced the event of interest prior to the start of the study.
\item Interval censoring where the only information is that the event occurs within some intervals.
\end{enumerate}
\end{itemize}
\end{frame}


\begin{frame}{Survival data}
\begin{itemize}
\item Left truncation occurs when the subjects have been at risk before entering the study (for example: life insurance policy holders where the study starts on
a fixed date, event of interest is age at death).
\item Right truncation occurs when the entire study population has already experienced the event of interest (for example: a historical survey of patients on a cancer registry).
\item Generally we deal with right censoring $\&$ sometimes left truncation.
\end{itemize}
\end{frame}

\section{Basic Quantities}
\begin{frame}{Functions}
\begin{itemize}
\item Let $X$ be the time until some specified event, and it's a non-negative random variable from a homogeneous population. 
\begin{enumerate}
\item The survival function: The probability of an individual surviving to time $x$. 
\item The hazard rate function: The chance an individual of age $x$ experiences the event in the next instant in time.
\item The probability density function: The unconditional probability of the event's occurring at time $x$. 
\item The mean residual life: The mean time to the event of the interest, given the event has not occurred at $x$.
\end{enumerate}
\item If we know any one of these four functions, then the other three can be uniquely determined.
\end{itemize}
\end{frame}

\begin{frame}{The survival function}
\begin{itemize}
\item The probability of an individual surviving beyond time x (experiencing the event after time x)
\begin{equation*}
S(x) = Pr(X>x) = 1-F(x) = \int_{x}^{\infty}f(t)dt
\end{equation*}
\item For a discrete random variable $X$.
\begin{equation*}
S(x) = Pr(X>x) = \sum_{x_j > x}^{}P(x_j)
\end{equation*}
\item For the Weibull distribution, $S(x) = exp(-\lambda x^{\alpha}),  \lambda>0,\alpha>0$, when $\alpha = 1$ is the exponential distribution.
\end{itemize}
\end{frame}

\begin{frame}{The hazard function}
\begin{itemize}
\item The hazard rate is defined by:
\begin{equation*}
h(x) = \lim_{\Delta x \to 0} \frac{P(x \le X < x+\Delta x | X \ge x)}{\Delta x}
\end{equation*}
\item If $X$ is a continuous random variable, $h(x)= \frac{f(x)}{S(x)} = -\frac{d \ln(S(x))}{dx}$
\begin{equation*}
H(x) = \int_{0}^{x} h(\mu) d\mu = -ln[S(x)] 
\end{equation*}
\begin{equation*}
S(x) = exp[-H(x)] = exp[-\int_{0}^{x}h(\mu)d\mu]
\end{equation*}
\item For the Weibull distribution, $\alpha > 1$, $h(x)$ increasing, $\alpha < 1$, $h(x)$ decreasing.
\end{itemize}
\end{frame}

\begin{frame}{The hazard function}
\begin{itemize}
\item If $X$ is a discrete random variable,
\begin{equation*}
h(x_j) = Pr(X = x_j | X \ge x_j) = \frac{P(x_j)}{S(x_{j-1})}, j=1,2,...
\end{equation*}
where $S(x_0)=1$. $P(x_j) = S(x_{j-1}) - S(x_j)$ and $h(x_j) = 1-S(x_j)/S(x_{j-1}), j=1,2,...$
\item So the survival function is:
\begin{equation*}
S(x) = \prod_{x_j \le x}^{} S(x_j)/S(x_{j-1}) = \prod_{x_j \le x}^{}[1-h(x_j)]
\end{equation*}
\end{itemize}
\end{frame}

\begin{frame}{The mean residual life function and median life}
\begin{itemize}
\item The mean residual life function measures their expected remaining lifetime(area under survival curve to the right of $x$ divided by $S(x)$)
\begin{equation*}
mrl(x) = E(X-x|X>x)
\end{equation*}
$\mu=mrl(0)$ is the mean life, total area under the survival curve.
\item For a continuous random variable,
\begin{equation*}
mrl(x) = \frac{\int_{x}^{\infty}(t-x)f(t)dt}{S(x)} = \frac{\int_{x}^{\infty}S(t)dt}{S(x)}
\end{equation*}
\begin{equation*}
\mu = E(x) = \int_{0}^{\infty}tf(t)dt = \int_{0}^{\infty}S(t)dt
\end{equation*}
\item The median lifetime for a continuous random variable $x$ is the $x_{0.5}$, so that $S(x_{0.5}) = 0.5$
\end{itemize}
\end{frame}

\section{Regression Models for Survival Data}
\begin{frame}{Models}
 \begin{itemize}
 \item Consider a failure time $X>0$, a vector $Z' = (Z_1,\cdots,Z_p)$ of explanatory variables associated with failure time $x$, $Z'$ may include quantitative variables (such as blood pressure, temperature, age, and weight), qualitative variables (such as gender, race, treatment, and disease status) and/or time-dependent variables,$Z'(x) = [Z_1(x), \cdots, Z_p(x)]$.
 \item Two approaches to the modeling of covariate effects on survival.
 \end{itemize}
\end{frame}

\begin{frame}{Models}
\begin{itemize}
\item 1. The first approach is analogous to the classical linear regression approach, $Y=\ln(X)$. A linear model is assumed for $Y$, namely,
\begin{equation*}
Y = \mu + \gamma' z +\delta w
\end{equation*}
where $\gamma' = (\gamma_1, \cdots, \gamma_p)$ is a vector of regression coefficients and $w$ is the error distribution.\linebreak
If $W \thicksim N(0,1)$, $Y \thicksim \log$ normal regression model.\linebreak
If $W\thicksim exp(w-e^{w}), -\infty < w < \infty$, $Y \thicksim$ Weibull regression model.\linebreak
This model is called the accelerated failure-time(AFT) model.
\end{itemize}
\end{frame}

\begin{frame}{The accelerated failure time model}
\begin{itemize}
\item Let $S_0(x)$ denote the survival function of $X=e^{Y}$ when $Z$ is 0, that is $S_0(x)$ is the survival function of $\exp(\mu +\delta w)$, and
\begin{equation*}
Pr(X>x|z) = S_0[x \exp(-\gamma' z)]	
\end{equation*}
\item So the hazard rate of an individual with a covariate value $Z$ for the model is related to a baseline hazard rate $h_0$ by
\begin{equation*}
h(x|z) = h_0[x\exp(-\gamma'z)]\exp(-\gamma'z)
\end{equation*}
\item Although the accelerated failure-time model provides a direct extension of the classical linear model's construction for explanatory variables for conventional data, for survival data, its use is restricted by the error distributions one can assume.
\end{itemize}
\end{frame}

\begin{frame}{Conditional hazard model}
\begin{itemize}
\item 2. The major approach to modeling the effects of covariates on survival is to model the conditional hazard rate as a function of the covariates.
\item Two general classes of models have been used to relate covariate effects to survival, the family of multiplicative hazard models and the family of additive hazard rate models.
\item For the family of multiplicative hazard rate models the conditional hazard rate of an individual with covariate vector $z$ is a product of a baseline hazard rate $h_0(x)$ and a non-negative function of the covariates, $c(\beta',z)$
\begin{equation*}
h(x|z) = h_0(x)c(\beta' z)
\end{equation*}
$h_0(x)$ may have a specified parametric form or it may be left as an arbitrary non-negative function. 
\end{itemize}
\end{frame}

\begin{frame}{Multiplicative hazards models}
\begin{itemize}
\item Most applications use the Cox model with $c(\beta' z)=exp(\beta' z)$,which is chosen for its simplicity and for the fact it is positive for any value of $\beta' z$.
\item A key feature of multiplicative hazards models is that, when all the covariates are fixed at time 0, the hazard rates of two individuals with distinct values of $z$ are proportional. 
\begin{equation*}
\frac{h(x|z_1)}{h(x|z_2)}= \frac{h_0(x)c(\beta'z_1)}{h_0(x)c(\beta'z_2)}=\frac{c(\beta'z_1)}{c(\beta'z_2)}
\end{equation*}
which is a constant independent of time.
\item So it can expressed in terms of a baseline survival function $S_0(x)$ as 
\begin{equation*}
S(x|z) = S_0(x)^{c(\beta'z)}
\end{equation*}
\end{itemize}
\end{frame}

\begin{frame}{Additive hazard rate models}
\begin{itemize}
\item The conditional hazard function is:
\begin{equation*}
h(x|z)=h_0(x)+\sum_{j=1}^{p}z_j(x)\beta_j(x)  
\end{equation*}
\item Estimation for additive models is typically made by non-parametric (weighted) least-squares methods. 
\end{itemize}
\end{frame}

\section{Models for Competing Risks}
\begin{frame}{Competing Risks}
\begin{itemize}
\item When each subject may fail due to one of $K(K
\ge 2)$ causes, called competing risks.
\item  Occurrence of one of these events precludes us from observing the other event on this patient. 
\item Another classical example of competing risks is cause-specific mortality, such as death from heart disease, death from cancer, death from other causes, etc.
\item Let $X_i$, $i=1, \cdots,K$ be the potential unobserv- able time to occurrence of the $i$th competing risk. What we observe for each patient is the time at which the subject fails from any cause, $T= min(X_1,\cdots,X_p)$ and an indicator $\delta$ which tells which of the $K$ risks caused the patient to fail, that is,$\delta = i$ if $T = X_i$
\end{itemize}
\end{frame}


\begin{frame}{Competing Risks}
\begin{itemize}
\item The basic competing risks parameter is the cause-specific hazard rate for risk $i$ defined by:
\begin{equation*}\label{eq1}
\begin{split}
h_i(t) &= \lim_{\Delta t \to 0} \frac{P[t \le T < t+\Delta t, \delta =i | T \ge t]}{\Delta t} \\
&=  \lim_{\Delta t \to 0} \frac{P[t \le T < t+\Delta t, \delta =i | x_j \ge t, j=1,2,\cdots,k]}{\Delta t} 
\end{split}
\end{equation*}
$h_i(t)$ tells us the rate at which subjects who have yet to experience any of the competing risks are experiencing the $i$th competing cause of failure. 
\item  The overall hazard rate of the time to failure, $T$,is the sum of these $K$ cause-specific hazard rates; that is
\begin{equation*}
h_T(t) = \sum_{i=1}^{K}h_i(t)
\end{equation*}
\end{itemize}
\end{frame}

\begin{frame}{Competing Risks}
\begin{itemize}
\item The cause-specific hazard rate can be derived from the joint survival function of the $K$ competing risks. Let $S(t_1,\cdots,t_K) = \linebreak Pr[x_1>t_1,\cdots,x_K>t_K]$.
\begin{equation*}
h_i(t) = \frac{-\partial S(t_1,\cdots,t_K)/\partial t_i |_{t_1 = \cdots = t_K =t} }{S(t,\cdots,t)}
\end{equation*}
\item In competing risks problems we are often interested not in the hazard rate but rather in some probability which summarizes our knowledge about the likelihood of the occurrence of a particular competing risk.Three probabilities are computed,crude, net, and partial crude probabilities.
\end{itemize}
\end{frame}

\begin{frame}{Competing Risks}
\begin{itemize}
\item The crude probability is the probability of death from a particular cause in the real world where all other risks are acting on the individual(e.g: P(death from heart disease) $\longrightarrow$ P(die from heart disease prior to age 50))
\item The net probability is the probability of death in a hypothetical world where the specific risk is the only risk acting on the population.(e.g: P(die from heart disease in the counterfactual world where men can only die from heart disease))
\item The partial crude probabilities are the probability of death in a hypothetical world where some risks of death have been eliminated.(e.g: P(dies from heart disease in a world where cancer has been cured)).
\end{itemize}
\end{frame}

\begin{frame}{Competing Risks}
\begin{itemize}
\item The Crude probabilities are typically expressed by the cause-specific sub-distribution function, also known as the cumulative incidence function, is defined as:
\begin{equation*}
F_i(t) = P[T\le t, \delta = i] = \int_{0}^{t}h_i(\mu)\exp\lbrace -H_T(\mu)\rbrace d\mu
\end{equation*}
Here $H_T(t) = \sum_{j=1}^{K}\int_{0}^{t}h_j(\mu)d\mu$ is the cumulative hazard rate of $T$.
\item The net survival function, $S_i(t)$, is the marginal survival function found from the joint survival function by taking $t_j=0$ for all $j\neq i$. When the competing risks are independent then the net survival function is related to the crude probabilities by:
\begin{equation*}
S_T(t) \le S_i(t) \le 1-F_i(t)
\end{equation*}
\end{itemize}
\end{frame}

\begin{frame}{Competing Risks}
\begin{itemize}
\item For partial crude probabilities, let $J$ be the set of causes that an individual can fail from and $J^C$ the set of causes which are eliminated from consideration.
\item Let $T^{J} = min (x_i, i \in J)$ then define the partial crude sub-distribution function by $F_{i}^{J}(t) = Pr[T^{J} \le t, \delta = i], i\in J$
\item Define a partial crude hazard rate:
\begin{equation*}
\lambda_{i}^{J}(t) = \frac{-\partial S(t_1,\cdots,t_K)/\partial t_i |_{t_j =t, t_j \in J; t_j = 0, t_j \in J^{C}} }{S(t_1,\cdots,t_p)|_{t_j =t, t_j \in J; t_j = 0, t_j \in J^{C}}}
\end{equation*}
\end{itemize}
\end{frame}
\end{document}
\documentclass{beamer}

% Some common packages
\usepackage{graphicx, color}
\usepackage{alltt}
\usepackage{booktabs, calc, rotating}
\usepackage{natbib}
\usepackage{multicol}
\usepackage{amsmath, amsbsy, amssymb, amsthm, graphicx}
\usepackage[english]{babel}
\usepackage{xkeyval} 
\usepackage{xfrac}
\usepackage[normalem]{ulem}
\usepackage{fancyvrb} 
\usepackage{tikz, geometry, tkz-graph, xcolor}
\usepackage[latin1]{inputenc}
\usepackage{times}
\usepackage[T1]{fontenc}

% Some shortcuts
\newcommand{\cov}{\mathrm{cov}}
\newcommand{\dif}{\mathrm{d}}
\newcommand{\bigbrk}{\vspace*{2in}}
\newcommand{\smallbrk}{\vspace*{.1in}}
\newcommand{\midbrk}{\vspace*{1in}}
\newcommand{\red}[1]{{\color{red}#1}}
\newcommand{\blue}[1]{{\color{blue}#1}}
\newcommand{\green}[1]{{\color{green}#1}}
\newcommand{\calc}[1]{{\fbox{\mbox{#1}}}}
\newcommand{\Var}{\mathrm{Var}}%
\newcommand{\Cov}{\mathrm{Cov}}%

\usetheme[numbering=fraction, progressbar = frametitle]{metropolis}
% section page?
% \usetheme[sectionpage=none]{metropolis}
% place progressbar under frametitle?
% \usetheme[progressbar = frametitle]{metropolis}



% To center title?
\setbeamertemplate{title page}[default]
% To add a rectangle box for title?
% \setbeamercolor{titlelike}{parent = frametitle}   

% item color and shape
% \setbeamertemplate{itemize item}{\color{mDarkTeal}$\bullet$}
% \setbeamertemplate{itemize subitem}{\color{mDarkTeal}$\bullet$}
% \setbeamertemplate{itemize subsubitem}{\color{mDarkTeal}$\bullet$}

% UTD colors
\definecolor{warmgray10}{RGB}{118,106,98}
\definecolor{warmgray2}{RGB}{213,210,202}
\definecolor{warmgray1}{RGB}{230,230,230}
\definecolor{flameOrange}{RGB}{199,91,18}
\definecolor{ecoGreen}{RGB}{0,133,86}

% only change item color 
\setbeamercolor*{itemize item}{fg = mDarkTeal}
\setbeamercolor*{itemize subitem}{fg = mDarkTeal}
\setbeamercolor*{itemize subsubitem}{fg = mDarkTeal}
% enumerate color
\setbeamercolor*{enumerate item}{fg = mDarkTeal}
\setbeamercolor*{enumerate subitem}{fg = mDarkTeal}
\setbeamercolor*{enumerate subsubitem}{fg = mDarkTeal}
% description color
\setbeamercolor*{description item}{fg = mDarkTeal}
\setbeamercolor*{description subitem}{fg = mDarkTeal}
\setbeamercolor*{description subsubitem}{fg = mDarkTeal}

% define blocks
\newenvironment<>{problock}[1]{
  \begin{actionenv}#2
    \def\insertblocktitle{#1}
    \par
    \mode<presentation>{
      \setbeamercolor{block title}{fg = white, bg = ecoGreen} %mDarkTeal}
      \setbeamercolor{block body}{fg = black, bg = warmgray1}
    }
    \usebeamertemplate{block begin}}
  {\par\usebeamertemplate{block end}\end{actionenv}}

\newenvironment<>{defblock}[1]{
  \begin{actionenv}#2
    \def\insertblocktitle{#1}
    \par
    \mode<presentation>{
      \setbeamercolor{block title}{fg = white,bg = flameOrange} %mDarkBrown}
      \setbeamercolor{block body}{fg = black,bg = warmgray1}
    }
    \usebeamertemplate{block begin}}
  {\par\usebeamertemplate{block end}\end{actionenv}}

% \setbeamerfont{block title}{size={}}
% \setbeamercolor{titlelike}{parent=structure,bg=flameOrange!80!black,fg=white} % title
\mode
<all>

% fancy for Verbatim?
\fvset{frame = single, framesep = 1mm,fontfamily = courier,
  fontsize = \scriptsize, numbers = left, framerule = .3mm,
  numbersep = 1mm,commandchars = \\\{\}}

% reset text width
\setbeamersize{text margin left = 5pt,text margin right = 5pt}

\title[Short title]{Statistics for Biology and Health\\ Chapter 4 Estimation of Basic Quantities for Other Sampling Schemes\\ 
}
\author[Author name]{Qi Guo}
% \institute[UTD]{Department of Mathematical Sciences \\ The University of Texas at Dallas}
\date{July 20,2019}
% UTD logo on title page
% \titlegraphic{\includegraphics[width=5cm]{mono_nsm_print_header.jpg}}
\titlegraphic{\includegraphics[scale = .27]{UTDmono_NSM_header_RGB-1.png}}

\begin{document}

% default title thicker line
% need to adjust
% \frame{
% \maketitle
% \begin{tikzpicture}[remember picture, overlay]
%   \def\xmargin{.1cm}     % Left and right margin of the title line
%   \def\yshift{-0.1cm}   % Moves the title line upwards (-1.25cm moves title line downwards)
%   \draw[mLightBrown,line width=2pt] (current page.west) ++(\xmargin,\yshift) -- ++({\paperwidth-2*\xmargin},0);
% \end{tikzpicture}
% }

\frame{
  \vspace{1cm}
  \maketitle
  \begin{tikzpicture}[remember picture, overlay]
    \def\xmargin{.8cm}     % Left and right margin of the title line
    \def\yshift{0.6cm}   % Moves the title line upwards 
    \draw[mLightBrown,line width=2.2pt] (current page.west) ++(\xmargin,\yshift) -- ++({\paperwidth-2*\xmargin},0);
  \end{tikzpicture}
}


% Set up UTD backgroud
% Show background from here on
\setbeamercolor*{item}{fg=red}
\bgroup
\usebackgroundtemplate{
  \tikz[overlay,remember picture] \node[opacity=0.05, at=(current page.center)] {
    \includegraphics[height=\paperheight,width=\paperwidth]{UTDbg}};}


\begin{frame}{Contents}
  \setbeamertemplate{section in toc}[sections numbered]
  \tableofcontents[hideallsubsections]
\end{frame}

\section{Introduction}
\begin{frame}{Introduction}
\begin{itemize}
\item Last Chapter we focus on estimate of the right-censored and left-truncated data, which is most common. 
\item In this chapter, we examine about left censoring, censored individuals provide information indicating only that the event has occurred prior to entry into the study.
\item Right-truncated data, which samples arise when one samples individuals from event records and, retrospectively determines the time to event.
\item Estimation techniques for grouped data.  
\end{itemize}
\end{frame}

\section{Estimation of the Survival Function for Left, Double, and Interval Censoring}
\begin{frame}{Estimation of Left Censoring Data}
\begin{itemize}
\item Examples of pure left censoring are rare. More common are samples which include both left and right censoring.
\item And in this case we use a modified KM estimator suggested by Turnbull, is based on an iterative procedure which extends the notion of a self-consistent estimator.
\item To construct the estimator we assume that there is a grid of time points $0 = t_0 < t_1 < t_2 < \cdots < t_m$ at which subjects are observed. 
\item Let $d_i$ be the number of deaths at time $t_i$, and $c_i$,$r_i$ are the number of individuals left-censored and right-censored at time $t_i$.
\end{itemize}
\end{frame}

\begin{frame}{Algorithm}
\begin{itemize}
\item The only information the left-censored observations at $t_i$ give us is that the event of interest has occurred at some $t_j \le t_i$
\item  The self-consistent estimator estimates the probability that this event occurred at each possible $t_j$ less than $t_i$ based on an initial estimate of the survival function. 
\item Using this estimate, we compute an expected number of deaths at $t_j$, which is then used to update the estimate of the survival function and the procedure is repeated until the estimated survival function stabilizes.
\end{itemize}
\end{frame}


\begin{frame}{Algorithm}
\begin{itemize}
\item \texttt{Step 0}: Produce an initial estimate of the survival function at each $t_j$,$S_0(t_j)$. Note any legitimate estimate will work. Turnbull?s suggestion is to use the KM estimate obtained by ignoring the left-censored observations.
\item \texttt{Step $(K+1)1$}: Using the current estimate of $S$, estimate $p_{ij} = p[t_{j-1} < x \le t_i | x\le t_i]$ by $\frac{S_K(t_{j-1}) - S_K(t_j)}{1-S_K(t_i)}$ for $j \le i $. 
\item \texttt{Step $(K+1)2$}: Using the results of the previous step, estimate the number of events at time $t_i$ by $\hat{d}_i = d_i +\sum_{i=j}^{m}c_i p_{ij}$.
\item \texttt{Step $(K+1)3$}: Compute the KM estimator based on the estimated right-censored data with $\hat{d}_i$ events and $r_i$ right-censored observations at $t_i$, ignoring the left-censored data. If this estimate, $S_{K+1}(t)$, is close to $S_K(t)$ for all $t_i$, stop the procedure; if not, go to step 1.
\end{itemize}
\end{frame}

\begin{frame}{Estimation of the interval-censored data}
\begin{itemize}
\item Sometimes data may be interval-censored. Here the only information we have for each individual is that their event time falls in an interval $(L_i, R_i]$,$i=1,...,n$, but the exact time is unknown.
\item And we still have a iteration like the estimarion of left-censored data in page 6.
\end{itemize}
\end{frame}

\section{Estimation of the Survival Function for Right-Truncated Data}
\begin{frame}{Right-Truncated Data}
\begin{itemize}
\item For right-truncated data, only individuals for which the event has occurred by a given date are included in the study. Right truncation arises commonly in the study of infectious diseases.
\item Let $T_i$ denote the chronological time at which the $i$th individual is infected and $X_i$ the time between infection and the onset of disease.
\item Sampling consists of observing $(T_i,X_i)$ for patients over the period $(0\ to\ \tau)$. Note that only patients who have the disease prior to $\tau$ are included in the study.
\item Estimation for this type of data proceeds by reversing the time axis. Let $R_i = \tau - X_i$. The $R_i$'s are now left-truncated in that only individuals with value of $T_i \le R_i$ are included in the sample. 
\end{itemize}
\end{frame}

\section{Estimation of Survival in the Cohort Life Table}
\begin{frame}{Estimation of Survival in the Cohort Life Table}
\begin{itemize}
\item A``cohort'' is a group of individuals who have some common origin from which the event time will be calculated
\item They are followed over time and their event time or censoring time is recorded to fall in one of $k+1$ adjacent, nonoverlapping intervals, $(a_{j-1}, a_j]$, $j=1,...,k+1$.
\item A traditional cohort life table presents the actual mortality experience of the cohort from the birth of each individual to the death of the last surviving member of the cohort.
\item The basic construction of the cohort life table is a little complicated, and more details in columns in P152.
\end{itemize}
\end{frame}
\end{document}
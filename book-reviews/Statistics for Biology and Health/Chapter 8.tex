\documentclass{beamer}

% Some common packages
\usepackage{graphicx, color}
\usepackage{alltt}
\usepackage{booktabs, calc, rotating}
\usepackage{natbib}
\usepackage{multicol}
\usepackage{amsmath, amsbsy, amssymb, amsthm, graphicx}
\usepackage[english]{babel}
\usepackage{xkeyval} 
\usepackage{xfrac}
\usepackage[normalem]{ulem}
\usepackage{fancyvrb} 
\usepackage{tikz, geometry, tkz-graph, xcolor}
\usepackage[latin1]{inputenc}
\usepackage{times}
\usepackage[T1]{fontenc}

% Some shortcuts
\newcommand{\cov}{\mathrm{cov}}
\newcommand{\dif}{\mathrm{d}}
\newcommand{\bigbrk}{\vspace*{2in}}
\newcommand{\smallbrk}{\vspace*{.1in}}
\newcommand{\midbrk}{\vspace*{1in}}
\newcommand{\red}[1]{{\color{red}#1}}
\newcommand{\blue}[1]{{\color{blue}#1}}
\newcommand{\green}[1]{{\color{green}#1}}
\newcommand{\calc}[1]{{\fbox{\mbox{#1}}}}
\newcommand{\Var}{\mathrm{Var}}%
\newcommand{\Cov}{\mathrm{Cov}}%

\usetheme[numbering=fraction, progressbar = frametitle]{metropolis}
% section page?
% \usetheme[sectionpage=none]{metropolis}
% place progressbar under frametitle?
% \usetheme[progressbar = frametitle]{metropolis}



% To center title?
\setbeamertemplate{title page}[default]
% To add a rectangle box for title?
% \setbeamercolor{titlelike}{parent = frametitle}   

% item color and shape
% \setbeamertemplate{itemize item}{\color{mDarkTeal}$\bullet$}
% \setbeamertemplate{itemize subitem}{\color{mDarkTeal}$\bullet$}
% \setbeamertemplate{itemize subsubitem}{\color{mDarkTeal}$\bullet$}

% UTD colors
\definecolor{warmgray10}{RGB}{118,106,98}
\definecolor{warmgray2}{RGB}{213,210,202}
\definecolor{warmgray1}{RGB}{230,230,230}
\definecolor{flameOrange}{RGB}{199,91,18}
\definecolor{ecoGreen}{RGB}{0,133,86}

% only change item color 
\setbeamercolor*{itemize item}{fg = mDarkTeal}
\setbeamercolor*{itemize subitem}{fg = mDarkTeal}
\setbeamercolor*{itemize subsubitem}{fg = mDarkTeal}
% enumerate color
\setbeamercolor*{enumerate item}{fg = mDarkTeal}
\setbeamercolor*{enumerate subitem}{fg = mDarkTeal}
\setbeamercolor*{enumerate subsubitem}{fg = mDarkTeal}
% description color
\setbeamercolor*{description item}{fg = mDarkTeal}
\setbeamercolor*{description subitem}{fg = mDarkTeal}
\setbeamercolor*{description subsubitem}{fg = mDarkTeal}

% define blocks
\newenvironment<>{problock}[1]{
  \begin{actionenv}#2
    \def\insertblocktitle{#1}
    \par
    \mode<presentation>{
      \setbeamercolor{block title}{fg = white, bg = ecoGreen} %mDarkTeal}
      \setbeamercolor{block body}{fg = black, bg = warmgray1}
    }
    \usebeamertemplate{block begin}}
  {\par\usebeamertemplate{block end}\end{actionenv}}

\newenvironment<>{defblock}[1]{
  \begin{actionenv}#2
    \def\insertblocktitle{#1}
    \par
    \mode<presentation>{
      \setbeamercolor{block title}{fg = white,bg = flameOrange} %mDarkBrown}
      \setbeamercolor{block body}{fg = black,bg = warmgray1}
    }
    \usebeamertemplate{block begin}}
  {\par\usebeamertemplate{block end}\end{actionenv}}

% \setbeamerfont{block title}{size={}}
% \setbeamercolor{titlelike}{parent=structure,bg=flameOrange!80!black,fg=white} % title
\mode
<all>

% fancy for Verbatim?
\fvset{frame = single, framesep = 1mm,fontfamily = courier,
  fontsize = \scriptsize, numbers = left, framerule = .3mm,
  numbersep = 1mm,commandchars = \\\{\}}

% reset text width
\setbeamersize{text margin left = 5pt,text margin right = 5pt}

\title[Short title]{Statistics for Biology and Health\\ Chapter 8 Refinements of the semiparametirc proportional hazards model\ 
}
\author[Author name]{Qi Guo}
% \institute[UTD]{Department of Mathematical Sciences \\ The University of Texas at Dallas}
\date{July 24,2019}
% UTD logo on title page
% \titlegraphic{\includegraphics[width=5cm]{mono_nsm_print_header.jpg}}
\titlegraphic{\includegraphics[scale = .27]{UTDmono_NSM_header_RGB-1.png}}

\begin{document}

% default title thicker line
% need to adjust
% \frame{
% \maketitle
% \begin{tikzpicture}[remember picture, overlay]
%   \def\xmargin{.1cm}     % Left and right margin of the title line
%   \def\yshift{-0.1cm}   % Moves the title line upwards (-1.25cm moves title line downwards)
%   \draw[mLightBrown,line width=2pt] (current page.west) ++(\xmargin,\yshift) -- ++({\paperwidth-2*\xmargin},0);
% \end{tikzpicture}
% }

\frame{
  \vspace{1cm}
  \maketitle
  \begin{tikzpicture}[remember picture, overlay]
    \def\xmargin{.8cm}     % Left and right margin of the title line
    \def\yshift{0.6cm}   % Moves the title line upwards 
    \draw[mLightBrown,line width=2.2pt] (current page.west) ++(\xmargin,\yshift) -- ++({\paperwidth-2*\xmargin},0);
  \end{tikzpicture}
}


% Set up UTD backgroud
% Show background from here on
\setbeamercolor*{item}{fg=red}
\bgroup
\usebackgroundtemplate{
  \tikz[overlay,remember picture] \node[opacity=0.05, at=(current page.center)] {
    \includegraphics[height=\paperheight,width=\paperwidth]{UTDbg}};}


\begin{frame}{Contents}
  \setbeamertemplate{section in toc}[sections numbered]
  \tableofcontents[hideallsubsections]
\end{frame}

\section{Introduction}
\begin{frame}{Introduction}
\begin{itemize}
\item  When the covariate may become a time-dependent variable, in notes before we introduce a $Z(t)$ instead of $Z$, and for commonly used model:
\begin{equation*}
h[t|Z(t)] = h_0(t)\exp[\beta' Z(t)] = h_0(t)\exp\big[\sum_{k=1}^{p}\beta_k Z_k(t) \big]
\end{equation*}
\item If the proportional hazard assumption is violated for a variable, we can stratify on this variable.Stratification fits a different baseline hazard function for each stratum, so that the form of the hazard function for different levels of this variable is not constrained by their hazards being proportional. 
\item And the basic proportional hazards model can be extended to left-truncated survival data. 
\end{itemize}
\end{frame}

\section{Time-Dependent Covariates}
\begin{frame}{Time-Dependent Covariates}
\begin{itemize}
\item  Now our data based on a sample of size $n$, consists of the triple $[T_j,\delta_j, [Z_j(t), 0 \le t \le T_j]], \ j = 1, \cdots,n$, where $T_j$ is the time on study for the $j$th patient, $\delta_j$ is the event indicator, and $Z_j(t) = [Z_{j1}(t), \cdots, Z_{jp}(t)]'$ is the vector of covariates for the $j$th individual.
\item And we assume that the censoring time and event are independent, the distinct event time $t_1 < t_2 < \cdots < t_D$, so $Z_{(i)}(t_i)$ is the covrariate associated with the individual whose failure time is $t_i$ and $R(t_i)$ is the risk set at time $t_i$, so the partial likelihood is given by:
\begin{equation*}
L(\beta) = \prod_{i=1}^{D}\frac{\exp \big[\sum_{b=1}^{p}\beta_b Z_{(i)b}(t_i)\big]}{\sum_{j\in R(t_i)}^{}\exp \big[\sum_{b=1}^{p}\beta_b Z_{jb}(t_i)\big]}
\end{equation*}
\end{itemize}
\end{frame}

\begin{frame}{Time-Dependent Covariates}
\begin{itemize}
\item In notes before we know there are a cut points here to coded the time-dependent variable into 0 and 1, if $t<$ time at which event occurs, let $Z(t) = 0$, and 1 o.w.
\item And we skip this part as it's pretty close as before.
\end{itemize}
\end{frame}

\section{Stratified Proportional Hazards Models}
\begin{frame}{Coding Covariates}
\begin{itemize}
\item Here the subjects in the $j$th stratum have a arbitrary baseline hazards function $h_{0j}(t)$ and the effect of other explanatory variables on the hazards function can be represented by a proportional hazards model in that stratum as :
\begin{equation*}
h_j[t|Z(t)] = h_{0j}(t)\exp[\beta' Z(t)],\ j = 1,\cdots,s
\end{equation*}
\item The regression coefficients are assumed to be the same in each stratum although the baseline hazard functions may be different and completely unrelated.
\end{itemize}
\end{frame}

\begin{frame}{Estimation and hypothesis testing}
\begin{itemize}
\item The partial log likelihood function is given by: 
\begin{equation*}
LL(\beta) = [LL_1(\beta)] +  [LL_2(\beta)] + \cdots +  [LL_s(\beta)]
\end{equation*}
where  $[LL_j(\beta)]$ is the log partial likelihood using only the data for those individuals in the $j$th stratum.
\item A key assumption in using a stratified proportional hazards model is that the covariates are acting similarly on the baseline hazard function in each stratum.
\item This can be tested by using either a likelihood ratio test or a Wald test.
\end{itemize}
\end{frame}

\begin{frame}{Estimation and hypothesis testing}
\begin{itemize}
\item Fit the stratified model, which assumes common $\beta'$s in each stratum, and obtain the log partial likelihood, $LL(b)$.
\item Using only data from the $j$th stratum, a Cox model is fit and the estimator $b_j$ and the log partial likelihood $LL_j(b_j)$are obtained. 
\item The log likelihood under the model, with distinct covariates for each of the $s$ strata, is $\sum_{j=1}^{s}LL_j(b_j)$.
\item The likelihood ratio chi square for the test that the $
beta'$s are the same in each stratum is $-2[LL(b) - \sum_{j=1}^{s}LL_j(b_j)]$ which has a large-sample, chi-square distribution with $(s-1)p$ degrees of freedom under the $H_0$. 
\end{itemize}
\end{frame}

\section{Left Truncation}
\begin{frame}{Introduction}
\begin{itemize}
\item Actually all of the material in this chapter is similar as before, and for the left truncated data, it arises when the event time $X$ is the age of the subject and persons are not observed from birth but rather from some other time $V$ corresponding to their entry into the study.
\item For example, the age, $X_i$, at death for the $i$th subject in a retirement center in California was recorded. Because an individual must survive to a sufficient age $V_i$ to enter the retirement community, and all individuals who died prior to entering the retirement community were not included in this study, the life lengths considered in this study are left-truncated.
\item Another situation is when the event time $X$ is measured from some landmark, but only subjects who experience some intermediate event at time $V$ are to be included in the study. The times $V$ are sometimes called delayed entry times.
\end{itemize}
\end{frame}

\begin{frame}{Formulate a proportional model}
\begin{itemize}
\item To formulate a proportional hazards regression model for a set of covariates $Z$, we model the conditional hazard rate of $t$, given $Z$ and $X>V$:
\begin{equation*}
h(t|Z, X>V) \cong \frac{P(X=t|Z, X>V)}{P(X\ge t|Z, X>V)}
\end{equation*}
\item If the event time $X$ and the entry time $V$ are conditionally independent, given the covariates $Z$, then $h(t|Z(t),X>V) = h(t|Z(t))$ 
\end{itemize}
\end{frame}

\begin{frame}{Estimate the coefficients}
\begin{itemize}
\item The partial likelihoods are modified to account for delayed entry into the risk set, and in notes before, we only need to add a delay entry time in R codes, so for the partial likelihoods will be different in the risk set $R(t)$.
\item Define the risk set $R(t)$ at time $t$ as the set of all individuals who are still under study at a time just prior to $t$. Here, $R(t) = \lbrace j|V_j <t < T_j \rbrace$, and the rest are same. 
\end{itemize}
\end{frame}


\end{document}